\documentclass[12 pt, latterpaper,twoside]{article}

\title{ASSIGNMENT NO.6}

\author{Md Umaid Shaan}

\usepackage{circuitikz}

\begin{document}


\maketitle


\section{The Boolean Expression For "e"}
\\ $e = A.\overline{B}.\overline{C}.\overline{D} + A.B.\overline{C}.\overline{D} + \overline{A}.\overline{B}.C.\overline{D}+A.\overline{B}.C\overline{D}+A.B.C.\overline{D}+A.\overline{B}.\overline{C}.D$
\section{K-map Expression (simplified)}
after simplifying the above boolean expression using k-maps it looks like this

$e= \overline{A}.B+A.\overline{D}+\overline{B}.C$

\subsection{The Combinational Circuit Using only NAND GATES}
\begin{figure}[h]
    \centering
    \begin{circuitikz}
\draw
(3,0.5)node[nand port](myand1){}
(3,-2)node[nand port](myand2){}
(3,-4.5)node[nand port](myand3){}
(0,0.2)node[nand port](mynand1){}
(0,-3.3)node[nand port](mynand2){}
(-2,0.2)node[left](D){D}
(-2,-3.3)node[left](B){B};
\begin{scope} 
\ctikzset{tripoles/american nand port/height=1.6}
\draw(6,-2)node[nand port, number inputs=3](myor1){};
\end{scope}
\draw
(myand3.in 2)node[left]{C}
(myand2.in 1)node[left]{A}
(myand1.out)--(myor1.in 1)
(myand2.out)--(myor1.in 2)
(myand3.out)--(myor1.in 3)
(myand1.in 1)node[left]{A}
(myor1.out)node[right]{$e$}
(D)--++(0:0.5)node[]{}--(mynand1.in 1)
(D)--++(0:0.5)node[]{}--(mynand1.in 2)
(B)--++(0:0.5)node[]{}--(mynand2.in 1)
(B)--++(0:0.5)node[]{}--(mynand2.in 2)
(mynand1.out)--(myand1.in 2)
(mynand2.out)--++(0:1)node[]{}--(myand2.in 2)
(mynand2.out)--++(0:1)node[]{}--(myand3.in 1)
;
\end{circuitikz}
    \caption{Circuit diagram using only nand gates}
    \label{fig:my_label}
\end{figure}


\end{document}
